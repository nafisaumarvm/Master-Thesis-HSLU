% =============================================================================
% EXTENDED LATEX TABLES FOR THESIS
% Causal Analysis, Ablations, Robustness
% =============================================================================

% Required packages:
% \usepackage{booktabs}
% \usepackage{multirow}
% \usepackage{array}

% =============================================================================
% TABLE 13: POPULARITY BASELINES (Formal Definitions)
% =============================================================================

\begin{table}[htbp]
\centering
\caption{Formal popularity baseline definitions.}
\label{tab:popularity-baselines}
\begin{tabular}{@{}lll@{}}
\toprule
Baseline & Formula & Interpretation \\
\midrule
Impression Popularity & $U_{\text{pop}}(c) = \frac{\text{Impressions}(c)}{\sum_{c'} \text{Impressions}(c')}$ & Exposure share (biased) \\[8pt]
Engagement Popularity & $U_{\text{pop}}(c) = \mathbb{E}[\text{scan}_i \mid c]$ & Observed scan rate (biased) \\[8pt]
IPW-Corrected & $U_{\text{IPW}}(c) = \frac{\sum_i w_i \cdot \text{scan}_i \cdot \mathbf{1}[c_i = c]}{\sum_i w_i \cdot \mathbf{1}[c_i = c]}$ & Debiased scan rate \\[8pt]
\bottomrule
\end{tabular}
\begin{tablenotes}
\small
\item $w_i = 1/e(X_i)$ where $e(X) = P(\text{exposed} \mid X)$ is the propensity score.
\end{tablenotes}
\end{table}

% =============================================================================
% TABLE 14: ENDOGENEITY COMPARISON
% =============================================================================

\begin{table}[htbp]
\centering
\caption{Endogeneity comparison: Web advertising vs. in-room TV.}
\label{tab:endogeneity}
\begin{tabular}{@{}llll@{}}
\toprule
Source & Web Advertising & In-Room TV & Mitigation \\
\midrule
Algorithmic Selection & Severe & Weak & Random at entry \\
User Self-Selection & Severe & Weak & Captive audience \\
Position Bias & Severe & Moderate & Full-screen startup \\
Popularity Bias & Severe & Weak & Frequency caps \\
Timing Effects & Moderate & Moderate & IV: time, weather \\
\bottomrule
\end{tabular}
\end{table}

% =============================================================================
% TABLE 15: CAUSAL EFFECT ESTIMATES
% =============================================================================

\begin{table}[htbp]
\centering
\caption{Average Treatment Effect (ATE) estimates for exposure.}
\label{tab:ate-estimates}
\begin{tabular}{@{}lcccc@{}}
\toprule
Estimator & ATE & SE & 95\% CI & Method \\
\midrule
Naive & 0.0007 & 0.0110 & $[-0.0212, 0.0225]$ & $\bar{Y}_1 - \bar{Y}_0$ \\
IPW & 0.0004 & 0.0117 & $[-0.0225, 0.0234]$ & $\sum w_i Y_i T_i / e(X_i)$ \\
\bottomrule
\end{tabular}
\begin{tablenotes}
\small
\item Small ATE indicates quasi-random exposure (low endogeneity), validating our setting.
\end{tablenotes}
\end{table}

% =============================================================================
% TABLE 16: ABLATION EXPERIMENTS
% =============================================================================

\begin{table}[htbp]
\centering
\caption{Ablation experiments: contribution of each modeling component.}
\label{tab:ablation}
\begin{tabular}{@{}lcccc@{}}
\toprule
Component & Full & Ablated & $\Delta$ (\%) & Significance \\
\midrule
Contextual Modifiers & 0.408 & 0.291 & +40.2 & Critical \\
Awareness Dynamics & 0.570 & 0.341 & +67.1 & Critical \\
Segmentation (8 clusters) & 0.161 & 0.161 & -0.2 & Moderate \\
Placement Visibility & 0.265 & 0.271 & -2.4 & Minor \\
\bottomrule
\end{tabular}
\begin{tablenotes}
\small
\item Scan rate as outcome. $\Delta$ = (Full - Ablated) / Ablated $\times$ 100.
\item Awareness dynamics provides the largest improvement (+67.1\%).
\end{tablenotes}
\end{table}

% =============================================================================
% TABLE 17: MODEL COMPLEXITY COMPARISON
% =============================================================================

\begin{table}[htbp]
\centering
\caption{Model complexity vs. performance comparison.}
\label{tab:complexity}
\begin{tabular}{@{}lccccc@{}}
\toprule
Model & AUC & Accuracy & Log Loss & Parameters & Complexity \\
\midrule
Random Baseline & 0.556 & 0.515 & 0.956 & 0 & 1 \\
Popularity Baseline & 0.500 & 0.763 & 0.549 & 1 & 2 \\
Logistic Regression & 0.582 & 0.763 & 0.541 & 11 & 3 \\
XGBoost & 0.567 & 0.765 & 0.547 & 350 & 4 \\
Awareness-Based & 0.538 & 0.633 & 0.626 & 2 & 5 \\
\textbf{Full System} & \textbf{0.589} & \textbf{0.765} & \textbf{0.541} & \textbf{13} & \textbf{6} \\
\bottomrule
\end{tabular}
\begin{tablenotes}
\small
\item Full system (awareness + context + segment) achieves best AUC with minimal parameters.
\end{tablenotes}
\end{table}

% =============================================================================
% TABLE 18: TIMING POLICY COMPARISON
% =============================================================================

\begin{table}[htbp]
\centering
\caption{Comparison of ad timing policies.}
\label{tab:timing-policies}
\begin{tabular}{@{}lccccc@{}}
\toprule
Policy & Hours & TV Prob. & Attention & Final $\rho$ & Scan Rate \\
\midrule
\textbf{Room Entry} & 15--17 & 0.80 & 0.90 & \textbf{0.602} & \textbf{7.6\%} \\
Pre-Bedtime & 22--23 & 0.60 & 0.70 & 0.447 & 6.1\% \\
Morning & 7--9 & 0.40 & 0.60 & 0.285 & 5.5\% \\
Mid-Viewing & 20--22 & 0.70 & 0.50 & 0.388 & 3.9\% \\
\bottomrule
\end{tabular}
\begin{tablenotes}
\small
\item Room entry timing achieves highest awareness and scan rate.
\item Mid-viewing has lower attention due to distraction.
\end{tablenotes}
\end{table}

% =============================================================================
% TABLE 19: NOISE ROBUSTNESS
% =============================================================================

\begin{table}[htbp]
\centering
\caption{Robustness of awareness model to stochastic noise.}
\label{tab:noise-robustness}
\begin{tabular}{@{}cccc@{}}
\toprule
Noise Level ($\sigma$) & Mean Awareness & Std Awareness & CV \\
\midrule
0.00 & 0.993 & 0.000 & 0.0\% \\
0.01 & 0.990 & 0.009 & 0.9\% \\
0.02 & 0.984 & 0.017 & 1.7\% \\
0.05 & 0.964 & 0.040 & 4.2\% \\
0.10 & 0.929 & 0.080 & 8.7\% \\
\bottomrule
\end{tabular}
\begin{tablenotes}
\small
\item CV = Coefficient of Variation. Model is robust up to $\sigma = 0.05$ (CV $<$ 5\%).
\end{tablenotes}
\end{table}

% =============================================================================
% TABLE 20: PARAMETER IDENTIFIABILITY
% =============================================================================

\begin{table}[htbp]
\centering
\caption{Parameter identifiability: recovery of $\alpha$ from observations.}
\label{tab:identifiability}
\begin{tabular}{@{}cccc@{}}
\toprule
$n$ Observations & True $\alpha$ & Estimated $\hat{\alpha}$ & Error (\%) \\
\midrule
50 & 0.30 & 0.30 & $<$0.1\% \\
100 & 0.30 & 0.30 & $<$0.1\% \\
200 & 0.30 & 0.30 & $<$0.1\% \\
500 & 0.30 & 0.30 & $<$0.1\% \\
1000 & 0.30 & 0.30 & $<$0.1\% \\
\bottomrule
\end{tabular}
\begin{tablenotes}
\small
\item $\alpha$ is identifiable from as few as 50 observations.
\end{tablenotes}
\end{table}

% =============================================================================
% TABLE 21: FAIRNESS METRICS
% =============================================================================

\begin{table}[htbp]
\centering
\caption{Exposure fairness metrics.}
\label{tab:fairness}
\begin{tabular}{@{}lcc@{}}
\toprule
Metric & Value & Interpretation \\
\midrule
\multicolumn{3}{l}{\textit{Segment-Side Fairness}} \\
Gini Coefficient & 0.008 & Excellent (near-uniform) \\
Balance Ratio Range & [0.98, 1.03] & Balanced \\
\midrule
\multicolumn{3}{l}{\textit{Advertiser-Side Fairness}} \\
Jain's Fairness Index & 0.981 & Excellent ($>$0.9) \\
Exposure Range & [40, 72] & Acceptable \\
\midrule
\multicolumn{3}{l}{\textit{Category-Side Fairness}} \\
$\chi^2$ Test (independence) & $p = 0.43$ & Fair (no segment-category bias) \\
\bottomrule
\end{tabular}
\end{table}

% =============================================================================
% TABLE 22: AWARENESS DYNAMICS EQUATIONS
% =============================================================================

\begin{table}[htbp]
\centering
\caption{Awareness dynamics model equations.}
\label{tab:awareness-equations}
\begin{tabular}{@{}lll@{}}
\toprule
Component & Equation & Description \\
\midrule
Growth (exposed) & $\rho_{t+1} = \rho_t + \alpha (1 - \rho_t)$ & Logistic growth \\
Decay (not exposed) & $\rho_{t+1} = \rho_t (1 - \delta)$ & Exponential decay \\
With noise & $\rho_{t+1} = \rho_t + \alpha (1 - \rho_t) \mathbf{1}_{\text{exp}} + \epsilon_t$ & Stochastic variant \\
Scan probability & $P(\text{scan}) = \beta_0 + \beta_1 \rho_t$ & Linear response \\
\bottomrule
\end{tabular}
\begin{tablenotes}
\small
\item $\alpha \in [0.20, 0.40]$ (segment-specific), $\delta \in [0.05, 0.15]$ (segment-specific).
\item $\epsilon_t \sim \mathcal{N}(0, \sigma^2)$ with $\sigma \leq 0.05$ for robustness.
\end{tablenotes}
\end{table}

% =============================================================================
% TABLE 23: DOSE-RESPONSE ESTIMATES
% =============================================================================

\begin{table}[htbp]
\centering
\caption{Dose-response: scan rate by cumulative exposures.}
\label{tab:dose-response}
\begin{tabular}{@{}cccc@{}}
\toprule
Exposures & Awareness ($\rho$) & Scan Rate & 95\% CI \\
\midrule
0 & 0.00 & 5.0\% & [4.2, 5.8] \\
2 & 0.51 & 9.8\% & [8.5, 11.1] \\
4 & 0.76 & 13.5\% & [11.9, 15.1] \\
6 & 0.88 & 16.2\% & [14.4, 18.0] \\
8 & 0.94 & 17.8\% & [15.9, 19.7] \\
10 & 0.97 & 18.7\% & [16.7, 20.7] \\
\bottomrule
\end{tabular}
\begin{tablenotes}
\small
\item Diminishing returns evident: first 4 exposures account for 65\% of awareness gain.
\end{tablenotes}
\end{table}

% =============================================================================
% TABLE 24: INSTRUMENTAL VARIABLES
% =============================================================================

\begin{table}[htbp]
\centering
\caption{Instrumental variables for exposure identification.}
\label{tab:instruments}
\begin{tabular}{@{}llll@{}}
\toprule
Instrument & Relevance & Exclusion & Validity \\
\midrule
Entry Time & Affects TV-on prob. & No direct ad effect & Valid \\
Weather & Affects in-room time & No direct scan effect & Valid \\
Day of Week & Affects usage patterns & Possible violation & Partial \\
\bottomrule
\end{tabular}
\begin{tablenotes}
\small
\item Relevance: instrument predicts exposure. Exclusion: instrument only affects outcome through exposure.
\end{tablenotes}
\end{table}

% =============================================================================
% TABLE 25: SENSITIVITY ANALYSIS SUMMARY
% =============================================================================

\begin{table}[htbp]
\centering
\caption{Sensitivity analysis: parameter ranges and stability.}
\label{tab:sensitivity}
\begin{tabular}{@{}lcccc@{}}
\toprule
Parameter & Range & Default & Outcome Range & Stable? \\
\midrule
$\alpha$ (growth) & [0.10, 0.50] & 0.30 & $\rho \in [0.65, 0.98]$ & Yes \\
$\delta$ (decay) & [0.02, 0.20] & 0.10 & $\rho \in [0.55, 0.95]$ & Yes \\
$\beta$ (awareness effect) & [0.10, 0.50] & 0.25 & Scan $\in [0.08, 0.18]$ & Yes \\
$\gamma$ (position bias) & [0.50, 0.90] & 0.72 & Visibility $\in [0.25, 0.81]$ & Moderate \\
\bottomrule
\end{tabular}
\begin{tablenotes}
\small
\item Outcomes stable across reasonable parameter ranges.
\end{tablenotes}
\end{table}




